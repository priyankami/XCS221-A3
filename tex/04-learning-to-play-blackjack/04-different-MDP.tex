\item \points{4d}
Sometimes, we might reasonably wonder how an optimal policy learned for one MDP
might perform if applied to another MDP with similar structure but slightly
different characteristics.  For example, imagine that you created an MDP to
choose an optimal strategy for playing "traditional" blackjack, with a standard
card deck and a threshold of 21.  You're living it up in Vegas every weekend,
but the casinos get wise to your approach and decide to make a change to the
game to disrupt your strategy: going forward, the threshold for the blackjack
tables is 17 instead of 21.  If you continued playing the modified game with
your original policy, how well would you do?  (This is just a hypothetical
example; we won't look specifically at the blackjack game in this problem.)

To explore this scenario, let's take a brief look at how a policy learned using
value iteration responds to a change in the rules of the MDP.

\begin{itemize}
  \item First, run value iteration on the |originalMDP| (defined for you in
  |submission.py|) to compute an optimal policy for that MDP.

  \item Next, simulate your policy on |newThresholdMDP| (also defined for you in
  |submission.py|) by calling |simulate| with an instance of |FixedRLAlgorithm|
  that has been instantiated using the policy you computed with value iteration.
  What is the expected reward from this simulation?

  {\em Hint: read the
  documentation (comments) for the |simulate| function in util.py, and look
  specifically at the format of the function's return value.}

  \item Now try simulating Q-learning directly on |newThresholdMDP| with
  |blackjackFeatureExtractor| and the default exploration probability. What is
  your expected reward under the new Q-learning policy?  Provide some
  explanation for how the rewards compare, and why they are different.
\end{itemize}
